\documentclass{article}
\usepackage[utf8]{inputenc}
\usepackage[letterpaper, margin=1in]{geometry}
\usepackage{amsmath}
\usepackage{amssymb }

\title{CS5820 HW9}
\author{Renhao Lu, NetID: rl839}
	
\begin{document}
	\maketitle
	
	\section{NP-Complete proof sketches}
	\subsection{UNSAT}
	The approach is not correct. Because UNSAT is not a NP problem. No matter which certificate we are provided, we can not prove that there is no other input that make $\Phi$ true. 
	\subsection{Magnet Problem}
	The approach is correct. The complete proof is in below:
	\subsubsection{Magnet problem is in NP}
	Certificate: a set of strings (may contain replicate strings)\\
	Check: if the set of strings used up all symbols in the symbol collection, the answer is yes. If the number of strings is $a$ and the number of symbols is $b$, the checking time is $n(ab)$, which is polynomial.
	\subsubsection{Reduce Hamilton Cycle problem to Magnet problem}
	\paragraph{Convert input}
	Given a directed graph $G=(V,E)$. For every node $v\in V$, create two letters $v_1,v_2$. For every edge $(u,v)\in E$, create a string $u_2v_1$.
	\paragraph{Convert output}
	If Magnet problem has a yes answer, Hamilton cycle problem has a yes answer. Otherwise, the answer of Hamilton cycle problem is no.
	\paragraph{Running time}
	Assume $|V|=n$, $|E|=m$. We will create $2n$ symbols and $m$ strings, and the length of all strings are $2$. Hence, the total running time is $O(n+m)$, which is polynomial.
	\paragraph{Correctness}
	We will prove that the yes/no answer of magnet problem is equivalent to the answer of Hamilton problem.\\
	If Hamilton problem has a solution, we note this cycle $C=\{(v_1,v_2),(v_2, v_3),...,(v_n,v_{n-1})\}$ where $V=\{v_1, v_2,...,v_n\}$. We know that for each node $v_i\in V$, exist exactly two edges $(v_i,v_j),(v_{j'},v_i)\in C$. Then we created symbols $v_{i1},v_{i2}$ and strings as we previously described. Here, for each edge $(v_i,v_j)\in C$, we add the string $v_{i2} v_{j1}$ into our collection. Because each node $v_i$, appeared twice in two different edge at left side and right side respectively, each symbol $v_{i1},v_{i2}$ must appear in the string collection exactly once. Meanwhile, all symbol must have been used because Hamilton cycle must travel throught all nodes. Hence, the string collection would be a solution for Magnet problem. \\
	If the Magnet problem has a solution, we note the alphabet as $A=\{v_{11}, v_{12}, v_{21}, v_{22},...,v_{n1},v_{n2}\}$ the solution string collection $S=\{v_{i2}v_{j1}, v_{i'2}v_{j'1},...\}$. For each symbol, we have exactly one "amgnet", so each symbol must appear exactly once in $S$. So we know for each $v_{i1} and v_{i2}$, exist one string $v_{j2}v_{i1}$ and $v_{i2}v_{j'1}$ in $S$. Hence, we pick one random symbol $v_{i_2}$ as the start point and reorginze $S$ in a end-to-end pattern as following: $\{v_{i2}v_{j1}, v_{j2}v_{k1},...,v_{p2}v_{t1},v_{t2}v_{i1}\}$. Then we create a cycle as for the solution for Hamilton cycle problem: $C=\{(v_i, v_j),(v_j,v_k),...,(v_p, v_t),(v_t,v_i)\}$. This cycle doesn't have any replicates because no replicates exists in $S$. Meanwhile $C$ covers all nodes because $S$ covers all symbols in $A$. Hence $C$ is a correct solution for Hamilton problem. 
	
	\subsection{Subway Riding}
	The apporach is not correct. From the porblem we can know that $\sum_{i=1}^n {w_i}\geq W$. Hence createing $n$ cycles will need totally $(\sum_{i=1}^n {w_i}) -n+1$ nodes in $G$. Creating new graph $G$ will take at least $O(W-n)$, which is not polynomial time.
	
	\section{Disrupting an Enemy's Railway Network II}
	Railway problem is in P, and it can be reduced to the original railway disruption problem. 
	\paragraph{Step 1: convert input}
	Given the input of the railway problem directed graph $G=(V,E)$, $h\in V$, $T\subseteq V$, and integer $k<|T|$. We create a new directed graph $G'=(V',E')$ for the input of original railway problem. We first add $h$ to $V'$. Then For each node $v_i\in V (v_i\neq h)$, we add two nodes $v_{i-in}, v_{i-out}$ into $V'$, and we add edges $(v_{i-in}, v_{i-out})$ to $E'$ with capacity $c_e=1$. For each edge $(v_i, v_j)\in E$, we add edge $(v_{i-out},v_{j-in})$ into $E'$ with capacity $c_e=+\infty$. For all $v_i\in T$, we add $v_i-out$ to $T'$. And we set all $w_i=+\infty$ for $v_{i-out}\in T'$. Then we pass the directed graph $G'$, the capacity of the edges $\{c_e\}$, set $T'$, and $\{w_i\}$ as the input of the oringinal railway problem. Because all $w_i=+\infty$, we only need to find the minimized $\sum_{c\in F}{c_e}$ in the original railway problem.
	\paragraph{Step 2: convert output}
	We note the output of the original railway problem is a set $F$ of edges to be destroyed. If $|F|\leq k$, the answer for the railway problem II is yes. Otherwise, the answer is no.
	\paragraph{Step 3: Running time}
	Assume for the input of railway problem II $G=(V,E)$, $|V|=n,|E|=m, |T|=p<n$. \\
	Converting input: in $G'=(V',E')$, $|V'|=2n-1$, $|E'|=m+n-1$, so building up the graph and setting the edge capacities will take $O(m+n)$. Initializing $|T'|$ and $\{w_i\}$ takes $O(p)<O(n)$, so toal would still be $O(m+n)$\\
	Converting output: Getting the output $F$ size can take at most $O(m+n)$, and comparing size with $k$ is $O(1)$. So total is $O(m+n)$\\
	Find the solution for the original railway problem I: The maximum possible flow in equal to $|T'|=|T|=p$. So the running time for the original railway problem is $O(|E'|p)=O((m+n)p)<O((m+n)n)$.\\
	The total running time of this reductions is $O(m+n)+O(m+n)+O((m+n)n)=O((m+n)n)$, which is polynomial.
	\paragraph{Step 3: Correctness}
	We claim that $|F|\leq k$ from the original railway problem is equivalent to the yes answer in the railway problem II.\\
	1. If $|F|\leq k$ from the original railway problem I, we can prove that railway problem II have a solution. We can first claim that All edges in $F$ is $(v_{i-in},v_{i-out})$ and doesn't contain any edge like $(v_{i-out},v_{j-in})$, because all edges $(v_{i-out},v_{j-in})$ has a capacity of $+\infty$, and they cannot be seperated by minimum s-t cut. Then we create a set of node $Q$, and add each $v_i$ if $(v_{i-in},v_{i-out})\in F$. We can tell that $|Q|=|F|\leq k$. Because $w_i=+\infty$, all terminal nodes must be disconnected. So $Q$ is a solution for railway problem II. \\
	2. If railway problem II has a solution, we can prove that the solution $|F|$ not larger than $k$. We note $Q=\{v_{i1},v_{i2},...,v_{ik}\}$ is a set of statations as the solution of railway problem II. Then we set $F_1=\{(v_{i1-in},v_{i1-out}),(v_{i1-in},v_{i1-out}),...,(v_{ik-in},v_{ik-out})\}$. Because in railway problem II, disabling the statations in $Q$ can disconnect all terminals in $T$, destroying edges in $F_1$ can disconnect all terminals in railway problem I as well. Because for all $(v_{i-in},v_{i-out}), c_e=1$, $|F|_{optimal}=min\{\sum_{c\in F}{c_e}\}\leq \sum_{c\in F_1}{c_e}=|F_1|=k$. Hence the optimal $|F|\leq k$,
	\section{Side gig problem}
	\subsection{Side gig problem is in NP}
	Certificate: a set of jobs you plan to do during the summber $J={i_1,i_2,i_3,...}$\\
	Check: If for any two $i,j\in J, D_i\cap D_j=\O$, and $\sum_{i\in J}{p_i}\geq C$, the answer is yes.
	\subsection{Reduce independent set problem to side gig problem}
	\paragraph{Convert input}
	Given the input of independent set problem, undirected graph $G=(V,E)$ and integer $k$. We assume $V={1,2,3,...,n},|V|=n$ and $E={e_1,e_2,...,e_m}, |E|=m>n$. Then we create $n$ jobs, each job $i'$ corresponds to a node $i\in V$, and each job has the same payment $p_i=1$. Then we set summer break as $m$ days. For each edge $e_p=(i,j)\in E$, we add day $p$ into $D_i$ and $D_j$. At last we set the credit card debit $C=k$. Now we can pass $n,m,{D_i}, C$ as the input of side gig problem.
	\paragraph{Convert output}
	If the answer of side gig problem is yes, the output of independent set is also yes. Otherwise, the output of independent set is no.
	\paragraph{Running time}
	Converting input: Creating new job set, takes $O(n)$. Creating m days of summer break takes $O(m)$. Setting up ${D_i}$ need to update 2 times for each edge, so takes $O(2m)=O(m)$. Hence, converting input totally take $O(n+m)$, which is polynomial. 
	\paragraph{Correctness}
	We claim that the answer of independent set problem is equivalent to the answer of side gig problem.\\
	1. If independent set problem has a solution, we note this solution as set $S={i_1, i_2,...,i_k}$, then we create a job set $P={i_1',i_2',...,i_k'}$. The total payment $\sum_{i\in P}{p_i}=k=C$, hence the credit card debit can be paied off. Meanwhile, because in the independent set $S$, no two node share one edge, we know that in $P$, no two jobs share a same day. Hence, we know set $P$ is a solution for the side gig problem. \\
	2. If the side gig problem has a solution, we note this soltion as a set of jobs $P={i_1',i_2',i_3',...,i_k',...}$. Becasue the debit $C=k$, $|P|\geq C=k$. We select the first $k$ members in $P$ and transfor the jobs into another set of node which the jobs are corresponding to, $S={i_1,i_2,...,i_k}, |S|=k$. We claim that $S$ is an independent set because if some two nodes $i,j\in S$ and $(i,j)=e_p\in E$, $D_i\cap D_j ={p}\neq \O$, which is a contradiction. Hence, $P$ is a solution for independent set problem. 


\end{document}